\section{Conclusion et travaux futurs}~\label{sec:conclu}
En conclusion, ce rapport valide que l'hypothèse que la force de contact latérale subie par les roues des \acp{VSS}.
Pour ce faire, une évaluation expérimentale a été conduite sur 10 vitesses angulaires différentes avec un \ac{VSS} de \SI{260}{\kg}.
Une limite fonctionnelle des modèles a été observée à pour des vitesses angulaires de plus de \SI{2.0}{\radian\per\second}.
Une analyse détaillée et plusieurs hypothèses pour expliquer ce phénomène ont été proposés.

En travaux futur, pousser ce protocole expérimental sur plusieurs vitesses longitudinales et plusieurs types de terrains sera pertinent.
Une analyse approfondie du point de cassure, où les forces calculées avec les modèles linéaires plafonnent permettra de mieux comprendre ce phénomène et son origine.
Par la suite, utiliser des modèles de force de contact linéaires pour construite des modèles dynamiques complets de \ac{VSS} sera utile pour améliorer la performance de la navigation autonome et des systèmes de prise de décision qui opèrent avec des véhicules hors route.
Enfin, ces modèles linéaires, qui impliquent peu de paramètres, sont idéaux pour faire de la modélisation adaptive pour permettre de réduire l'erreur de modélisation quand le véhicule subit des changements majeurs de conditions de traction.
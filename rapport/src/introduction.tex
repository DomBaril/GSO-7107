\section{Introduction}~\label{sec:intro}
Un modèle de mouvement est un outil essentiel pour plusieurs opérations qui impliquent l'utilisation d'une flotte de véhicules.
En premier lieu, un modèle de mouvement permet d'analyser les forces en jeu dans l'opération d'un véhicule et permet d'analyser l'impact de plusieurs opérations réalisées sur une flotte de véhicules.
Cette analyse d'impact permet de guider plusieurs décisions et donc d'optimiser les coûts financiers et écologiques de la flotte.
En second lieu, un modèle de mouvement est un fondement critique dans l'automatisation de la conduite de véhicules. 
Cette tâche permettrait éventuellement d'accomplir plusieurs tâches redondantes sans nécessiter la présence d'un opérateur, ce qui pourrait supporter plusieurs industries qui souffrent de la pénurie de main d'oeuvre actuelle.
% Trouver 1 ou 2 citations pour statistiques
Ces modèles peuvent prendre deux formes : cinématiques et dynamiques.
Les premiers permettent de modéliser la géométrie du mouvement des véhicules et les deuxièmes permettent de modéliser les forces agissant sur les roues des véhicules.
Les modèles dynamiques sont plus riches et permettent de modéliser une plus grande plage de conditions auxquelles les véhicules sont soumis. 

Le fondement le plus important d'un modèle de mouvement de véhicule terrestre est la géométrie de direction. 
La géométrie dite \textit{Ackerman} est la plus répandue pour les véhicules routiers.
Toutefois, dans le cas de la conduite hors route, la géométrie \textit{skid-steer} est également très populaire.
Pour tourner ce type de véhicule, une différence de vitesse de rotation entre les roues de chaque côté du véhicule est imposée. 
Les \ac{VSS} offrent plusieurs avantages, notamment la simplicité mécanique, la robustesse, la manoeuvrabilité hors route et la capacité de tourner sur place~\citep{Shamah2001}.
Toutefois, ces véhicules sont soumis à une friction latérale élevée puisque les roues sont constamment parallèles à la direction longitudinale du véhicule. 
Comme les modèles de force de contact linéaires sont populaires pour les véhicules \textit{Ackerman}, il serait pertinent d'évaluer leur performance pour les \ac{VSS}, comme ceux-ci sont soumis à des forces latérales élevées.

Les contributions liées à ce rapport sont donc :
\begin{itemize}
	\item La formulation d'un modèle dynamique planaire de \ac{VSS};
	\item L'évaluation de l'exactitude d'un modèle de force de contact linéaire pour les forces latérales subies par le robot.
\end{itemize}
La structure du rapport est comme suit :
Une revue de la littérature connexe à ce travail est présentée dans la~\autoref{sec:revue}.
La~\autoref{sec:metho} explique la méthodologie utilisée pour atteindre les contributions de ce travail.
Les résultats sont présentés dans la~\autoref{sec:resultats}.
Une analyse détaillée des résultats suit à la~\autoref{sec:analyse}.
Enfin, une conclusion et les travaux futurs sont présentés dans la~\autoref{sec:conclu}.
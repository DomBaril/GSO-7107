\section{Revue de littérature}~\label{sec:revue}
Afin de montrer la pertinence de ce travail dans la littérature, nous montront les techniques de modélisation du mouvement pour les \ac{VSS} les plus populaires dans la littérature dans la~\autoref{sec:revue_planaires}. 
Ensuite, nous présentons les travaux pertinents par rapport aux forces de contact dans la~\autoref{sec:revue_contact}.

\subsection{Modèles de mouvement}~\label{sec:revue_planaires}
La modélisation du mouvement des \ac{VSS} est un problème actuel dans la littérature scientifique.
En raison de leur simplicité et de leur résilience à l'identification de paramètres éronnés, les modèles cinématiques empiriques sont actuellement les plus populaires, tel que celui présenté par~\citet{Mandow2007}.
Alternativement, le modèle présenté par~\citet{Seegmiller2014} permet de modéliser le comportement dynamique du robot comme des perturbations subies par un modèle cinématique simple connaît également de la popularité dans la littérature.

Bien que simples, ces modèles font l'hypothèse que le robot navigue sur un terrain plat et dur, ce qui est généralement faux dans le cadre de la conduite hors route.
\citet{Rabiee2019} ont introduit un modèle croisé entre la cinématique et la dynamique.
Dans ce modèle, l'accélération des roues sont utilisées pour minimiser le système d'équations de la dynamique du corps du \ac{VSS}.
Une fois le système d'équations minimisé, les valeurs sont introduites dans un modèle cinématique analogique à~\citep{Mandow2007} pour prédire le déplacement du véhicule.
Alternativement, \citet{Seegmiller2016} ont présenté une formulation générale permettant de modéliser le mouvement de tout types de véhicules en minimisant le calcul et en offrant une grande précision.

\subsection{Forces de contact}~\label{sec:revue_contact}
Tous les modèles dynamiques de véhicules nécessitent de modéliser les forces de contact des roues des véhicules.
Il existe un grand nombre de ce genre de modèles dans la littérature ainsi qu'une grande variété dans leur complexité.
Le modèle le plus simple est le modèle linéaire, qui assume une relation linéaire entre la vitesse du point de contact et la force de friction des roues~\citep{Pacejka2012}. 
Un second modèle populaire est appelé la \textit{Magic Formula}, proposé par~\citet{Pacejka2012}.
Ce modèle empirique permet de modéliser la perte de friction causée par le passage de friction statique à friction dynamique et est très populaire pour les véhicules routiers~\citep{Brach2011}.
Une adaptation pour un \ac{VSS} a également été proposée par~\citet{Maclaurin2011}.
Dans ce travail, l'auteur évalue les paramètres du modèle empirique en fonction de propriétés physiques d'un \ac{VSS} lourd.

Des modèles plus complexes basés sur l'interaction entre les roues et le sol ont aussi été proposés.
La majorité de ces modèles sont basés sur des travaux pionniers de~\citet{Wong1967}.
\citet{Ishigami2007} ont proposé un modèle de force de contact permettant de modéliser les forces latérales. 

Bien que plusieurs modèles complexes ont été proposés pour modéliser les forces de contact, les résultats montrés par~\citet{Seegmiller2016} montrent que les modèles linéaires offrent une erreur de modélisation inférieure aux autres modèles sur une longue trajectoire.
Notre hypothèse est que le nombre élevé de paramètres utilisés dans ces modèles affecte la capacité aux modèles de généraliser sur un grand nombre de conditions différentes.
Dans ce rapport, nous visons à évaluer l'exactitude des modèles de force de contact linéaires pour modéliser le mouvement des \acp{VSS}.
